\documentclass{report}
\usepackage{hyperref}
\begin{document}

\chapter*{Preface}

The work of Richard M. Stallman literally speaks for itself. From the documented source code to the published papers to the recorded speeches, few people have expressed as much willingness to lay their thoughts and their work on the line.
Such openness-if one can pardon a momentary un-Stallman adjective-is refreshing. After all, we live in a society that treats information, especially personal information, as a valuable commodity. The question quickly arises. Why would anybody want to part with so much information and yet appear to demand nothing in return?

As we shall see in later chapters, Stallman does not part with his words or his work altruistically. Every program, speech, and on-the-record bon mot comes with a price, albeit not the kind of price most people are used to paying.

I bring this up not as a warning, but as an admission. As a person who has spent the last year digging up facts on Stallman's personal history, it's more than a little intimidating going up against the Stallman oeuvre. "Never pick a fight with a man who buys his ink by the barrel," goes the old Mark Twain adage. In the case of Stallman, never attempt the definitive biography of a man who trusts his every thought to the public record.

For the readers who have decided to trust a few hours of their time to exploring this book, I can confidently state that there are facts and quotes in here that one won't find in any Slashdot story or Google search. Gaining access to these facts involves paying a price, however. In the case of the book version, you can pay for these facts the traditional manner, i.e., by purchasing the book. In the case of the electronic versions, you can pay for these facts in the free software manner. Thanks to the folks at O'Reilly \& Associates, this book is being distributed under the GNU Free Documentation License, meaning you can help to improve the work or create a personalized version and release that version under the same license.

If you are reading an electronic version and prefer to accept the latter payment option, that is, if you want to improve or expand this book for future readers, I welcome your input. Starting in June, 2002, I will be publishing a bare bones HTML version of the book on the web site, \url{http://www.faifzilla.org}. My aim is to update it regularly and expand the \emph{Free as in Freedom} story as events warrant. If you choose to take the latter course, please review Appendix C of this book. It provides a copy of your rights under the GNU Free Documentation License.

For those who just plan to sit back and read, online or elsewhere, I consider your attention an equally valuable form of payment. Don't be surprised, though, if you, too, find yourself looking for other ways to reward the good will that made this work possible.

One final note: this is a work of journalism, but it is also a work of technical documentation. In the process of writing and editing this book, the editors and I have weighed the comments and factual input of various participants in the story, including Richard Stallman himself. We realize there are many technical details in this story that may benefit from additional or refined information. As this book is released under the GFDL, we are accepting patches just like we would with any free software program. Accepted changes will be posted electronically and will eventually be incorporated into future printed versions of this work. If you would like to contribute to the further improvement of this book, you can reach me at \href{mailto:sam@inow.com}{sam@inow.com}.

\end{document}
